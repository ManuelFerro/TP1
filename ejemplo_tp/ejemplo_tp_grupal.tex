\documentclass[10pt,a4paper]{article}

\usepackage[spanish,activeacute,es-tabla]{babel}
\usepackage[utf8]{inputenc}
\usepackage{ifthen}
\usepackage{listings}
\usepackage{dsfont}
\usepackage{subcaption}
\usepackage{amsmath}
\usepackage[strict]{changepage}
\usepackage[top=1cm,bottom=2cm,left=1cm,right=1cm]{geometry}%
\usepackage{color}%
\newcommand{\tocarEspacios}{%
	\addtolength{\leftskip}{3em}%
	\setlength{\parindent}{0em}%
}

% Especificacion de procs

\newcommand{\In}{\textsf{in }}
\newcommand{\Out}{\textsf{out }}
\newcommand{\Inout}{\textsf{inout }}

\newcommand{\encabezadoDeProc}[4]{%
	% Ponemos la palabrita problema en tt
	%  \noindent%
	{\normalfont\bfseries\ttfamily proc}%
	% Ponemos el nombre del problema
	\ %
	{\normalfont\ttfamily #2}%
	\
	% Ponemos los parametros
	(#3)%
	\ifthenelse{\equal{#4}{}}{}{%
		% Por ultimo, va el tipo del resultado
		\ : #4}
}

\newenvironment{proc}[4][res]{%
	
	% El parametro 1 (opcional) es el nombre del resultado
	% El parametro 2 es el nombre del problema
	% El parametro 3 son los parametros
	% El parametro 4 es el tipo del resultado
	% Preambulo del ambiente problema
	% Tenemos que definir los comandos requiere, asegura, modifica y aux
	\newcommand{\requiere}[2][]{%
		{\normalfont\bfseries\ttfamily requiere}%
		\ifthenelse{\equal{##1}{}}{}{\ {\normalfont\ttfamily ##1} :}\ %
		\{\ensuremath{##2}\}%
		{\normalfont\bfseries\,\par}%
	}
	\newcommand{\asegura}[2][]{%
		{\normalfont\bfseries\ttfamily asegura}%
		\ifthenelse{\equal{##1}{}}{}{\ {\normalfont\ttfamily ##1} :}\
		\{\ensuremath{##2}\}%
		{\normalfont\bfseries\,\par}%
	}
	\renewcommand{\aux}[4]{%
		{\normalfont\bfseries\ttfamily aux\ }%
		{\normalfont\ttfamily ##1}%
		\ifthenelse{\equal{##2}{}}{}{\ (##2)}\ : ##3\, = \ensuremath{##4}%
		{\normalfont\bfseries\,;\par}%
	}
	\renewcommand{\pred}[3]{%
		{\normalfont\bfseries\ttfamily pred }%
		{\normalfont\ttfamily ##1}%
		\ifthenelse{\equal{##2}{}}{}{\ (##2) }%
		\{%
		\begin{adjustwidth}{+5em}{}
			\ensuremath{##3}
		\end{adjustwidth}
		\}%
		{\normalfont\bfseries\,\par}%
	}
	
	\newcommand{\res}{#1}
	\vspace{1ex}
	\noindent
	\encabezadoDeProc{#1}{#2}{#3}{#4}
	% Abrimos la llave
	\par%
	\tocarEspacios
}
{
	% Cerramos la llave
	\vspace{1ex}
}

\newcommand{\aux}[4]{%
	{\normalfont\bfseries\ttfamily\noindent aux\ }%
	{\normalfont\ttfamily #1}%
	\ifthenelse{\equal{#2}{}}{}{\ (#2)}\ : #3\, = \ensuremath{#4}%
	{\normalfont\bfseries\,;\par}%
}

\newcommand{\pred}[3]{%
	{\normalfont\bfseries\ttfamily\noindent pred }%
	{\normalfont\ttfamily #1}%
	\ifthenelse{\equal{#2}{}}{}{\ (#2) }%
	\{%
	\begin{adjustwidth}{+2em}{}
		\ensuremath{#3}
	\end{adjustwidth}
	\}%
	{\normalfont\bfseries\,\par}%
}

% Tipos

\newcommand{\nat}{\ensuremath{\mathds{N}}}
\newcommand{\ent}{\ensuremath{\mathds{Z}}}
\newcommand{\float}{\ensuremath{\mathds{R}}}
\newcommand{\bool}{\ensuremath{\mathsf{Bool}}}
\newcommand{\cha}{\ensuremath{\mathsf{Char}}}
\newcommand{\str}{\ensuremath{\mathsf{String}}}

% Logica

\newcommand{\True}{\ensuremath{\mathrm{true}}}
\newcommand{\False}{\ensuremath{\mathrm{false}}}
\newcommand{\Then}{\ensuremath{\rightarrow}}
\newcommand{\Iff}{\ensuremath{\leftrightarrow}}
\newcommand{\implica}{\ensuremath{\longrightarrow}}
\newcommand{\IfThenElse}[3]{\ensuremath{\mathsf{if}\ #1\ \mathsf{then}\ #2\ \mathsf{else}\ #3\ \mathsf{fi}}}
\newcommand{\yLuego}{\land _L}
\newcommand{\oLuego}{\lor _L}
\newcommand{\implicaLuego}{\implica _L}

\newcommand{\cuantificador}[5]{%
	\ensuremath{(#2 #3: #4)\ (%
		\ifthenelse{\equal{#1}{unalinea}}{
			#5
		}{
			$ % exiting math mode
			\begin{adjustwidth}{+2em}{}
				$#5$%
			\end{adjustwidth}%
			$ % entering math mode
		}
		)}
}

\newcommand{\existe}[4][]{%
	\cuantificador{#1}{\exists}{#2}{#3}{#4}
}
\newcommand{\paraTodo}[4][]{%
	\cuantificador{#1}{\forall}{#2}{#3}{#4}
}

%listas

\newcommand{\TLista}[1]{\ensuremath{seq \langle #1\rangle}}
\newcommand{\lvacia}{\ensuremath{[\ ]}}
\newcommand{\lv}{\ensuremath{[\ ]}}
\newcommand{\longitud}[1]{\ensuremath{|#1|}}
\newcommand{\cons}[1]{\ensuremath{\mathsf{addFirst}}(#1)}
\newcommand{\indice}[1]{\ensuremath{\mathsf{indice}}(#1)}
\newcommand{\conc}[1]{\ensuremath{\mathsf{concat}}(#1)}
\newcommand{\cab}[1]{\ensuremath{\mathsf{head}}(#1)}
\newcommand{\cola}[1]{\ensuremath{\mathsf{tail}}(#1)}
\newcommand{\sub}[1]{\ensuremath{\mathsf{subseq}}(#1)}
\newcommand{\en}[1]{\ensuremath{\mathsf{en}}(#1)}
\newcommand{\cuenta}[2]{\mathsf{cuenta}\ensuremath{(#1, #2)}}
\newcommand{\suma}[1]{\mathsf{suma}(#1)}
\newcommand{\twodots}{\ensuremath{\mathrm{..}}}
\newcommand{\masmas}{\ensuremath{++}}
\newcommand{\matriz}[1]{\TLista{\TLista{#1}}}
\newcommand{\seqchar}{\TLista{\cha}}

\renewcommand{\lstlistingname}{Código}
\lstset{% general command to set parameter(s)
	language=Java,
	morekeywords={endif, endwhile, skip},
	basewidth={0.47em,0.40em},
	columns=fixed, fontadjust, resetmargins, xrightmargin=5pt, xleftmargin=15pt,
	flexiblecolumns=false, tabsize=4, breaklines, breakatwhitespace=false, extendedchars=true,
	numbers=left, numberstyle=\tiny, stepnumber=1, numbersep=9pt,
	frame=l, framesep=3pt,
	captionpos=b,
}

\usepackage{caratula} % Version modificada para usar las macros de algo1 de ~> https://github.com/bcardiff/dc-tex


\titulo{Trabajo práctico 1: Especificación y WP}
\subtitulo{Fondo monetario común}

\fecha{\today}

\materia{Algoritmos y Estructuras de Datos (ex-algo2)}
\grupo{Grupo --}

\integrante{Chiquinho, Santiago}{490/23}{santichiquinho@gmail.com}
\integrante{Brucellaria, Tomas}{287/23}{tomasbrus54@gmail.com}
\integrante{Caetano, Franco}{351/23}{francocaetano20@gmail.com}
\integrante{Ferro, Manuel}{405/23}{manu.ferro.13@gmail.com}
% Pongan cuantos integrantes quieran

% Declaramos donde van a estar las figuras
% No es obligatorio, pero suele ser comodo
\graphicspath{{../static/}}

\begin{document}

\maketitle

\section{Especificación}

\subsection{Problema 1:}

\begin{proc}{redistribucionDeLosFrutos:}
	{\In recursos : \TLista{\ent}\nat, \In cooperan : \TLista{\bool}}{\TLista{\ent}}

	\requiere{
		\longitud[recursos] = \longitud[cooperan] \yLuego

		todoPositivo[recursos]
	}
	\asegura{
		reparticiones[recursos, cooperan] \yLuego

		todoIgualSiCooperan[recursos, cooperan]
	}

	\pred{todoPositivo}{recursos}
	{\paraTodo[unalinea]{i}{\ent}{
		0 $\leq$ i $<$ \longitud[recursos] \implicaLuego recursos[i] $>$ 0
	}}

	\pred{reparticiones}{recursos, cooperan}
	{\paraTodo[unaLinea]{i}{\ent}{
		(0 $\leq$ i $<$ \longitud[recursos] \implicaLuego ((cooperan[i] = \True \yLuego res[i] = FMC(recursos, cooperan)) \oLuego (cooperan[i] = \False \yLuego res[i] = recursos[i] + FMC(recursos, cooperan))))}
	}

	\pred{todoIgualSiCooperan}{recursos, cooperan}
	{\paraTodo[unaLinea]{i,j}{\ent}{
		0 $\leq$ i,j $<$ \longitud[recursos] \yLuego cooperan[i] = cooperan[j] /implicaLuego recursos[i] = recursos[j]}
	}

	\aux{FMC}{recursos, cooperan}{\nat}
	{\paraTodo[unaLinea]{i}{\ent}{
		0 $\leq$ i $<$ \longitud[recursos] \implicaLuego ( res = (\sum\limits_{n=0}^-{\longitud[recursos] - 1} (if cooperan[i] = \True then recursos[i] else 0 fi) / (\longitud[cooperan] - 1)))}
	}

\end{proc}

\subsection{Problema 2:}

\begin{proc}{trayectoriaDeLosFrutosIndividualesALargoPlazo}
	{\Inout trayectorias: \TLista{\TLista{\float}}, \In cooperan: \TLista{\bool}, \In apuestas: \TLista{\TLista{\float}},
	 \In pagos: \TLista{\TLista{\float}}, \In eventos: \TLista{\TLista{\nat}}}

	\requiere{
		\longitud[trayectorias] = \longitud[cooperan] = \longitud[apuestas] = \longitud[pagos] = \longitud[eventos] \yLuego %chequeo si las listas trabajan el mismo grupo de individuos

		mismoGrupoDeEventos(apuestas, pagos) \yLuego %chequeo si ambas refieren al mismo grupo de posibles eventos

		igualCantidadDeEventosPorIndividuo(eventos) \yLuego %chequeo que cada individuo haya vivido la misma cantidad de eventos

		todoPositivo(apuestas, pagos) \yLuego %chequeo que todos los pagos sean positivos, igual que todas las apuestas

		todoElRecursoFueApostado(apuestas) \yLuego %chequeo que las apuestas sumadas sean el 100% de los recursos del individuo

		trayectoriasInicialesCorrectas(trayectorias) %chequeo que trayectorias esta en "estado incial", y que son todas positivas
		}
 
	\asegura{
		PagoApuestaCorrecto(trayectorias, apuestas, pagos, eventos) \yLuego
		PagoPozoCorrecto(trayectorias, cooperan) \yLuego
		\paraTodo[unalinea]{i}{\ent}{
			0 $\leq$ i $<$ \longitud[eventos] \implicaLuego \longitud[trayectorias[i]] = \longitud[eventos[i]] + 1
		 	%chequeo que al final cada trayectoria mida 1 más que la lista de eventos
		}}

	\pred{todoPositivo}{apuestas, pagos}
	{\paraTodo[unalinea]{i,j}{\ent}{
		(0 $\leq$ i $<$ \longitud[apuestas] \yLuego 0 $\leq$ j $<$ apuestas[i]) \implicaLuego (apuestas[i][j] $>$ 0 \yLuego pagos[i][j] $>$ 0)}
	}
	
	\pred{trayectoriasInicialesCorrectas}{trayectorias}
	{\paraTodo[unalinea]{i}{\ent}{
		(0 $\leq$ i $<$ \longitud[trayectorias]) \implicaLuego ((\longitud[trayectorias[i]] = 1) \yLuego ((\cab)[trayectorias[i]] $>$ 0))}
	} 
	
	\pred{PagoApuestaCorrecto}{trayectorias, apuestas, pagos, eventos}
		{\paraTodo[unalinea]{i,j}{\ent}{
			((0 $\leq$ i $<$ \longitud[trayectorias]) \yLuego (1 $\leq$ j $<$ \longitud[trayectorias[i]])) \implicaLuego %acomodo los índices
	 		(trayectorias[i][j] = trayectorias[i][j - 1]*apuestas[i][eventos[i][j - 1]]*pagos[i][eventos[i][j - 1]])
	 	}
		}%chequeo que el recurso del i-esimo jugador a tiempo j sea su anterior multiplicado por la apuesta y el pago que correspondan

	\pred{PagoPozoCorrecto}{trayectorias, cooperan}
		{\paraTodo[unalinea]{i,j}{\ent}{ ((0 $\leq$ i $<$ \longitud[trayectorias]) \yLuego (1 $\leq$ j $<$ \longitud[trayectorias[i]])) \implicaLuego
			\IfThenElse{cooperan[i] = True}
				{trayectorias[i][j] = redistribucionDeLosFrutos (recursosATiempoJ(trayectorias, j)[i], cooperan)[i]}
				{trayectorias[i][j] = redistribucionDeLosFrutos (recursosATiempoJ(trayectorias, j)[i], cooperan)[i] + trayectorias[i][j - 1]}
		}
		}%chequeo que el recurso del i-esimo jugador a tiempo j se corresponda con su pago del pozo

	\aux{recursosATiempoJ}{trayectorias, j}{\TLista{\float}}
	{\paraTodo[unalinea]{i}{\ent}{
		(0 $\leq$ i $<$ \longitud[trayectorias]) \implicaLuego 
		(res[i] = trayectorias[i][j]) 
	}}%arma una lista de los recursos de los jugadores a tiempo j

\end{proc}

\subsection{Problema 3:}

\begin{proc}{trayectoriaExtrañaEscalera:}{\In trayectoria : \TLista{\float}}{\bool}

	\requiere{\longitud[trayectoria] $\geq$ 3} %la función de longitud no printea bien, preguntar
	\asegura{\res = \True \Iff (( \exists ! i \in[1,\longitud[trayectoria]-2]): (trayectoria[i] > trayectoria [i-1] \yLuego trayectoria [i] > trayectoria [i+1] ))}
\end{proc}

\subsection{Problema 4:}

\begin{proc}{individuoDecideSiCooperarONo:}{\In individuo : \nat, \In recursos : \TLista{\ent} ,\Inout cooperan : \TLista{\bool}, \In apuestas : \TLista{\TLista{\ent}}, \In pagos : \TLista{\TLista{\ent}}, \In eventos : \TLista{\TLista{\ent}}}

	\requiere{
		(\longitud[recursos]=\longitud[cooperan]=\longitud[apuestas]=\longitud[pagos]=\longitud[eventos]) \yLuego (0 $leq$ individuo $<$ \longitud[cooperan]) \yLuego 
		((\paraTodo[unalinea]{i,j,k,l,m}{\ent}){((0 $\leq$ i $<$ \longitud[recursos] \implicaLuego recursos[i] $>$ 0) \yLuego (0 $\leq$ j,m $<$ \longitud[pagos] \implicaLuego pagos[j][m] $>$ 0) \yLuego (0 $\leq$ k,l $<$ \longitud[apuestas] \implicaLuego apuestas[k][l] $>$ 0))}) %ver como printea el () después del paratodo
		}
	\asegura{
		(evaluarRecursosIndividuo(individuo, apuestas, pagos, recursos, eventos) \yLuego ((\ifthenelse{(\exists i \in)}{evaluarDesertores(recursos, cooperan, apuestas, pagos, eventos)}{\skip}) \yLuego \ifthenelse{test}{evaluarCooperadores(recursos, cooperan, apuestas, pagos, eventos)}{\skip}) \implicaLuego evaluarCooperacion(individuo, recursos, cooperan, apuestas, pagos, eventos)) %preguntar la estructura el existe
		}

	\aux{evaluarRecursosIndividuo:}{\In individuo : \nat, \In recursos : \TLista{\ent} , \In apuestas : \TLista{\TLista{\ent}}, \In pagos : \TLista{\TLista{\ent}}, \In eventos : \TLista{\TLista{\ent}}}{\float}
		\requiere{
		(\paraTodo[unalinea]{i}{\ent}{0 $\leq$ i $<$ \longitud[recursos] \yLuego (\longitud[pagos[i]]==\longitud[apuestas[i]])})
		}
		\asegura{\paraTodo[unalinea]{j}{\ent}{((0 $\leq$ j $<$ \longitud[eventos[individuo]]) \implicaLuego (res = recursos[individuo] + apuestas[individuo][eventos[individuo][j]] * pagos[individuo][eventos[individuo][j]] - apuestas[individuo][eventos[individuo][j]]))} %¿res?
		}
	

	\aux{evaluarDesertores:}{\In recursos : \TLista{\float} ,\In cooperan : \TLista{\bool}, \In apuestas : \TLista{\TLista{\ent}}, \In pagos : \TLista{\TLista{\ent}}, \In eventos : \TLista{\TLista{\ent}}}{\float} 
		\requiere{True}

		\asegura{
			(\paraTodo[unalinea]{i}{\ent}{0 $\leq$ i $<$ \longitud[recursos] \yLuego cooperan[i]=False} \implicaLuego evaluarRecursosDesertores(i) \and  ((\sum\limits_{i=0}^{\longitud[recursos] - 1}[recursos][i])/(\sum\limits_{i=0}^{\longitud[cooperan] - 1}(\ifthenelse{cooperan[i]=False}{1}{0}))))
		}
		
		\aux{evaluarRecursosDesertores}{\Inout recursos : \TLista{\ent} ,\In cooperan : \TLista{\bool}, \In apuestas : \TLista{\TLista{\ent}}, \In pagos : \TLista{\TLista{\ent}}, \In eventos : \TLista{\TLista{\ent}}}
			\requiere{
				(\paraTodo[unalinea]{i}{\ent}{0 $\leq$ i $<$ \longitud[recursos] \yLuego (\longitud[pagos[i]]==\longitud[apuestas[i]])})
			}

				\asegura{
					(\paraTodo[unalinea]{i,j}{\ent}{(((0 $\leq$ i $<$ \longitud[cooperan] \yLuego cooperan[i]=False)\yLuego (0 $\leq$ j $<$ \longitud[eventos[i]])) \implicaLuego (recursos[i]= recursos[i] - apuestas[i][eventos[i][j]] + apuestas[i][eventos[i][j]]*pagos[i][eventos[i][j]]))})
				}

	\aux{evaluarCooperadores:}{\In recursos : \TLista{\float} ,\In cooperan : \TLista{\bool}, \In apuestas : \TLista{\TLista{\ent}}, \In pagos : \TLista{\TLista{\ent}}, \In eventos : \TLista{\TLista{\ent}}}{\float} 
		\requiere{True}

		\asegura{
			(\paraTodo[unalinea]{i}{\ent}{0 $\leq$ i $<$ \longitud[recursos] \yLuego cooperan[i]=True} \implicaLuego evaluarRecursosCooperadores(i) \and  ((\sum\limits_{i=0}^{\longitud[recursos] - 1}[recursos][i])/(\sum\limits_{i=0}^{\longitud[cooperan] - 1}(\ifthenelse{cooperan[i]=True}{1}{0}))))
		}
		
		\aux{evaluarRecursosCooperadores}{\Inout recursos : \TLista{\ent} ,\In cooperan : \TLista{\bool}, \In apuestas : \TLista{\TLista{\ent}}, \In pagos : \TLista{\TLista{\ent}}, \In eventos : \TLista{\TLista{\ent}}} %Recursos pasaría a ser una lista de float. Tengo/puedo cambiarlo en la especificación?
			\requiere{
				(\paraTodo[unalinea]{i}{\ent}{0 $\leq$ i $<$ \longitud[recursos] \yLuego (\longitud[pagos[i]]==\longitud[apuestas[i]])})
			}

			\asegura{
				(\paraTodo[unalinea]{i,j}{\ent}{(((0 $\leq$ i $<$ \longitud[cooperan] \yLuego cooperan[i]=True)\yLuego (0 $\leq$ j $<$ \longitud[eventos[i]])) \implicaLuego (recursos[i]= recursos[i] - apuestas[i][eventos[i][j]] + apuestas[i][eventos[i][j]]*pagos[i][eventos[i][j]]))})
			}
		
	\aux{evaluarCooperación}{\In individuo : \nat, \In recursos : \TLista{\ent} ,\Inout cooperan : \TLista{\bool}, \In apuestas : \TLista{\TLista{\ent}}, \In pagos : \TLista{\TLista{\ent}}, \In eventos : \TLista{\TLista{\ent}}}
		\requiere{True}

			\asegura{
				(((evaluarRecursosIndividuo $\leq$ evaluarDesertores \y evaluarDesertores $>$ evaluarCooperadores)\implicaLuego (cooperan[individuo]= False)) \oLuego ((evaluarRecursosIndividuo $\leq$ evaluarCooperadores \y evaluarCooperadores $>$ evaluarDesertores) \implicaLuego (cooperan[individuo]=True)) \oLuego ((evaluarDesertores == evaluarRecursosCooperadores) \implicaLuego cooperan[individuo] = cooperan[individuo]))
			}

\end{proc}

\subsection{Problema 5:}

\begin{proc}{individuoActualizaApuesta:}
	{\In individuo: \nat, \In recursos: \TLista{\float}, \In cooperan: \TLista{\bool}, \Inout apuestas: \TLista{\TLista{\float}},
	 \In pagos: \TLista{\TLista{\float}}, \In eventos: \TLista{\TLista{\nat}}}

	\requiere{
		\longitud[recursos] = \longitud[cooperan] = \longitud[apuestas] = \longitud[pagos] = \longitud[eventos] \yLuego

		mismoGrupoDeEventos(apuestas, pagos) \yLuego

		todoPositivo(recursos, apuestas, pagos) \yLuego

		todoElRecursoFueApostado(apuestas) \yLuego	

		0 $\leq$ individuo $<$ \longitud[apuestas]}
		
	\asegura{%¿habrá que chequear que solo cambió apuestas[individuo]? ¿cómo se haría eso?

		\longitud[apuestas] = \longitud[pagos] \yLuego %chequeo que la lista de apuestas conserve la longitud

		%la longitud podría chequearse con pagos, eventos, cooperan o recursos,
		%¿es correcto exigir que se chequee con pagos? sino,
		%¿habría que hacer un auxiliar para que se permita chequear la longitud con cualquiera? preguntar
		
		mismoGrupoDeEventos(apuestas, pagos) \yLuego %chequeo que todas las apuestas (en particular apuestas[individuo]) conserven la longitud

		todoPositivo(recursos, apuestas, pagos) \yLuego %reciclo el pred ya definido, pero este chequea la condición también en recursos y pagos,
														%¿es correcto? ¿o deberìa chequearse sólo apuestas?
		
		todoElRecursoFueApostado(apuestas) \yLuego %chequeo que todas las apuestas (en particular apuestas[individuo]) sigan sumando 1
		 
		RecursosFinales(individuo, recursos, apuestas, pagos, eventos) $\geq$ RecursosFinales(individuo, recursos, apuestas, pagos, eventos)
		%la idea es chequear que los recursos finales individuales que las nuevas apuestas generan sean "los máximos"
		%¿Cómo hacerlo?
		
		}%se supone que apuestas ya está modificado, sino es así preguntar cómo se indica eso

	\pred{todoPositivo}{recursos, apuestas, pagos}
	{\paraTodo[unalinea]{i,j}{\ent}{
		(0 $\leq$ i $<$ \longitud[apuestas] \yLuego 0 $\leq$ j $<$ apuestas[i]) \implicaLuego
		(apuestas[i][j] $>$ 0 \yLuego pagos[i][j] $>$ 0 \yLuego recursos[i] $>$ 0)}
	}

	\aux{RecursosFinales}{individuo, recursos, apuestas, pagos, eventos}{\float}
	{recursos[individuo]*(\prod_{n=0}^{\longitud[eventos[individuo]] - 1} [pagos][individuo][eventos[individuo][n]]*[apuestas][individuo][eventos[individuo][n]])}
	%una productoria que calcula todo el proceso multiplicativo

\end{proc}

\subsection{Auxiliares de Especificación:}

\aux{todoElRecursoFueApostado}{apuestas}{\ent}
	{\paraTodo[unalinea]{i,j}{\ent}{
		(0 $\leq$ i $<$ \longitud[apuestas] \yLuego 0 $\leq$ j $<$ apuestas[i]) \implicaLuego (\sum\limits_{n=0}^{j} [apuestas][i][n])}
	}

\pred{mismoGrupoDeEventos}{apuestas, pagos}
	{\paraTodo[unalinea]{i}{\ent}{
		0 $\leq$ i $<$ \longitud[pagos] \implicaLuego \longitud[apuestas][i] = \longitud[pagos][i]}
	}
\pred{igualCantidadDeEventosPorIndividuo}{eventos}
	{\paraTodo[unalinea]{i,j}{\ent}{
		0 $\leq$ i,j $<$ \longitud[eventos] \implicaLuego \longitud[eventos][i] = \longitud[eventos][j]}
	}

\section{Demostraciones de correctitud}

 Se tiene la siguiente implementación:

\begin{lstlisting}
	res := recurso;
	i := 0;
	while (i < eventos.size()) do
		if eventos[i] then
			res := (res * apuesta.c) * pago.c;
		else
			res := (res * apuesta.s) * pago.s;
		endif
		i := i + 1
	endwhile
\end{lstlisting}

 Y se quiere demostrar que es correcta respecto de la siguiente especificación:
\begin{proc}{frutoDelTrabajoPuramenteIndividual}{\In recurso: \float, \In apuesta: , \In pago: , \In eventos: \TLista{\bool} }{\float}

	\requiere{apuesta.c + apuesta.s = 1 \Y pago.c $>$ 0 \Y pago.s $>$ 0 \Y apuesta.c $>$ 0 \Y apuesta.s $>$ 0 \Y recurso $>$ 0}

	\asegura{res = recurso * (apuesta.c * pago.c)^{#(eventos, T)} * (apuesta.s * pago.s)^{#(eventos, F)}}

	%\aux{#}{eventos, \bool}{\ent}{expresion}
\end{proc}

 Donde #apariciones(eventos,T) es el auxiliar utilizado en la teórica, y #(eventos,T) es su abreviación.
 Aclarar que se considera que $eventos[i] = True$ equivale a que salió cara.

\vspace{0.3cm}

 Para ello, se argumentará con el Teorema del Invariante y el Teorema de Terminación de Ciclos, cuyas hipotesis afirman que
 para ciertos "(Pc,Qc)" requiere y asegura, para cierto "while(B)do S" ciclo, "I" predicado y "fv" función variante, si se llegan a cumplir las siguientes propiedades:

\begin{enumerate} \setlength\itemsep{1cm}
	\item Pc \implica I

	\item {I \Y B}S{I}

	\item (I \Y \neg B) \implica Qc %chequear si false B printea bien 

	\item {(I \Y B) \Y V0 = fv}S{fv $<$ V0}

	\item I \Y fv $\leq$ 0 \implica \neg B
\end{enumerate}

 Entonces I es un invariante del ciclo, se cumple la poscondición a la salida del ciclo,
 la función variante es estrictamente decreciente y si alcanza la cota inferior la guarda ya no se cumple.
 Luego el ciclo es correcto respeto de la especificación y su ejecución siempre termina.
 Se proponen el siguiente Invariante y función variable:

\begin{equation}
	I: 0 $\leq$ i $\leq$ \longitud{eventos} \yLuego res = recurso * \prod\limits_{j=0}^{i-1} \IfThenElse{eventos[j]}{apuesta.c * pago.c}{apuesta.s * pago.s}
\end{equation}

\begin{equation}
	fv: \longitud{eventos} - i
\end{equation}

 Para los datos:

\begin{itemize}
	\item Pc : res = recurso \Y i = 0 \Y Req
	\item Qc : res = recurso * (apuesta.c * pago.c)^{#(eventos, T)} * (apuesta.s * pago.s)^{#(eventos, F)}
	\item B : i $<$ \longitud{eventos}
	\item S : el ciclo descrito en la implementación.
\end{itemize}

 Se comprueba, primero, si el Pc elegido es correcto para la especificación:

\begin{equation}
	(\equiv)
\end{equation}

 Se comprueba si los dados I y fv cumplen las hipotesis:

\subsection{Pc \implica I}

\begin{equation} %quizás las ecuaciones no están bien identadas, y por eso puede que no compile
\begin{split} %//poner el requiere acá//
	& Pc \implica I \\
	& (\equiv) res = recurso \Y i = 0 \Y Req  \implica \\
	& 0 $\leq$ i $\leq$ \longitud{eventos} \yLuego
	  res = recurso * \prod\limits_{j=0}^{i-1} \IfThenElse{eventos[j]}{apuesta.c * pago.c}{apuesta.s * pago.s}\\

	& (\equiv) res = recurso \Y i = 0 \Y Req \implica 
	  0 $\leq$ 0 $\leq$ \longitud{eventos} \yLuego res = recurso * \prod\limits_{j=0}^{0-1} \IfThenElse{eventos[j]}{apuesta.c * pago.c}{apuesta.s * pago.s}\\

	& (\equiv) res = recurso \Y i = 0 \Y Req \implica 
	  true \yLuego res = recurso * 1
\end{split} %//poner el requiere acá//
\end{equation}

 Luego, se comprueba que la precondición es más fuerte que el invariante.

\subsection{{I \Y B}S{I}}
 La tripla de Hoare {I \Y B}S{I} es válida si y sólo si la expresión $I \Y B \implica wp(S,I)$ es verdadera.

\vspace{0.3cm}

 Donde $wp(S,I)$ equivale a $wp(if...; i:= i + 1, I)$, con "if..." representando la elección booleana que ocurre dentro del ciclo.
 Luego, queda: $wp(if..., I^{i}_{i+1})$, que equivale a:

\begin{equation}
def(B_{if}) \yLuego ( B_{if} \Y wp(res:= res*apuesta.c*pago.c , I^{i}_{i+1})) \lor (\neg B_{if} \Y wp(res:= res*apuesta.s*pago.s , I^{i}_{i+1})) \equiv def(B_{if}) \yLuego ("wp1" \lor "wp2")
\end{equation}

 Con $B_{if} \equiv eventos[i]$ la guarda del if dentro del ciclo. 
 Observar que B_{if} es la misma guarda dentro de la productoria,
 por lo tanto si separamos en casos cuando $B_{if} \lor \neg B_{if}$,
 se puede sacar de la productoria el caso i-ésimo dependiendo de en que caso estemos.

\vspace{0.3cm}
 
 (Analizando ambos casos, omitiremos la indexación de i del invariante, que al reemplazar por i+1 es igual a (-1 $\leq$ i $\leq$ \longitud{eventos} - 1))
 Observando "wp1":

\begin{equation}
	res * (apuesta.c * pago.c) = recurso * \prod\limits_{j=0}^{i} \IfThenElse{eventos[j]}{apuesta.c * pago.c}{apuesta.s * pago.s} = \prod\limits_{j=0}^{i-1} \IfThenElse{eventos[j]}{apuesta.c * pago.c}{apuesta.s * pago.s} * (apuesta.c * pago.c) 
\end{equation}

 Como apuesta.c y pago.c no pueden ser nulos (el requiere de la especificación admite apuestas y pagos mayores a cero), podemos pasar dividiendo y queda:

\begin{equation}
	res = recurso * \prod\limits_{j=0}^{i-1} \IfThenElse{eventos[j]}{apuesta.c * pago.c}{apuesta.s * pago.s}
\end{equation}

 Análogamente, como apuesta.s y pago.s son no nulos, "wp2" equivale a:

\begin{equation}
	res * (apuesta.s * pago.s) = recurso * \prod\limits_{j=0}^{i} \IfThenElse{eventos[j]}{apuesta.c * pago.c}{apuesta.s * pago.s} = \prod\limits_{j=0}^{i-1} \IfThenElse{eventos[j]}{apuesta.c * pago.c}{apuesta.s * pago.s} * (apuesta.s * pago.s) 
\end{equation}

\begin{equation}
	res = recurso * \prod\limits_{j=0}^{i-1} \IfThenElse{eventos[j]}{apuesta.c * pago.c}{apuesta.s * pago.s}
\end{equation}

 Entonces $wp(S,I)$ equivale a:

\begin{equation}
	0 $\leq$ i $<$ \longitud{eventos} \yLuego (( B_{if} \Y -1 $\leq$ i $\leq$ \longitud{eventos} - 1 \Y res = recurso * \prod\limits_{j=0}^{i-1} \IfThenElse{eventos[j]}{apuesta.c * pago.c}{apuesta.s * pago.s} ) \lor ( \neg B_{if} \Y -1 $\leq$ i $\leq$ \longitud{eventos} - 1 \Y res = recurso * \prod\limits_{j=0}^{i-1} \IfThenElse{eventos[j]}{apuesta.c * pago.c}{apuesta.s * pago.s} )) (\equiv) 
\end{equation}
 
 Notar que dejamos de omitir la indexación de i del invariante, y $def(B_{if}) = 0 $\leq$ i $<$ \longitud{eventos}$
 Simplificando la ecuación, nos queda:

\begin{equation}
	0 $\leq$ i $<$ \longitud{eventos} \Y res = recurso * \prod\limits_{j=0}^{i-1} \IfThenElse{eventos[j]}{apuesta.c * pago.c}{apuesta.s * pago.s} \yLuego (eventos[i] = True \lor eventos[i] = False)
\end{equation}

\begin{equation}
	0 $\leq$ i $<$ \longitud{eventos} \Y res = recurso * \prod\limits_{j=0}^{i-1} \IfThenElse{eventos[j]}{apuesta.c * pago.c}{apuesta.s * pago.s}
\end{equation}

 Observando la expresión $I \Y B$, recordando que $B \equiv i $<$ \longitud{eventos}$ se obtiene que equivale a:

\begin{equation}
	0 $\leq$ i $<$ \longitud{eventos} \yLuego res = recurso * \prod\limits_{j=0}^{i-1} \IfThenElse{eventos[j]}{apuesta.c * pago.c}{apuesta.s * pago.s}
\end{equation}

 Por lo tanto nos queda la misma ecuación en ambos lados de la implicación:

\begin{equation}
	$I \Y B \implica wp(S,I)$
\end{equation}

\begin{equation}
	0 $\leq$ i $<$ \longitud{eventos} \yLuego res = recurso * \prod\limits_{j=0}^{i-1} \IfThenElse{eventos[j]}{apuesta.c * pago.c}{apuesta.s * pago.s} \implica 0 $\leq$ i $<$ \longitud{eventos} \yLuego res = recurso * \prod\limits_{j=0}^{i-1} \IfThenElse{eventos[j]}{apuesta.c * pago.c}{apuesta.s * pago.s}
\end{equation}

 Y terminamos de confirmar que $I$ es un invariante durante del ciclo.

\subsection{(I \Y \neg B) \implica Qc}

 Como la expresión $\neg B$ equivale a $i $\geq$ \longitud{eventos}$, la expresión $I \Y \neg B$ equivale a:
 
\begin{equation}
\begin{split}
	& $\leq$ i = \longitud{eventos} \yLuego
	  res = recurso * \prod\limits_{j=0}^{i-1} \IfThenElse{eventos[j]}{apuesta.c * pago.c}{apuesta.s * pago.s}\\

	& (\equiv) res = recurso * \prod\limits_{j=0}^{\longitud{eventos}-1} \IfThenElse{eventos[j]}{apuesta.c * pago.c}{apuesta.s * pago.s}\\
\end{split}
\end{equation}

 Observar que j indexa perfectamente a $eventos$. La productoria multiplica el término (apuesta.c * pago.c) las veces que se cumpla $ eventos[j] = True$,
 es decir, el término (apuesta.c * pago.c) aparece multiplicando exactamente #(eventos, T). Luego,
 (apuesta.c * pago.c)^{#(eventos, T)} es una parte de la productoria.

\vspace{0.3cm}
  
 Analogamente, (apuesta.s * pago.s)^{#(eventos, F)} es una parte de la productoria. como la productoria
 opera con un if cuya guarda es booleana, los términos de la misma solo se comportan de 2 maneras, las descritas anteriormente.
 Luego:

\begin{equation}
\begin{split}
	& res = recurso * \prod\limits_{j=0}^{\longitud{eventos}-1} \IfThenElse{eventos[j]}{apuesta.c * pago.c}{apuesta.s * pago.s}\\
	& (\equiv) res = recurso * (apuesta.c * pago.c)^{#(eventos, T)} * (apuesta.s * pago.s)^{#(eventos, F)}
\end{split}
\end{equation}

 Al ser la segunda expresión exactamente Qc, se concluye que (I \Y \neg B) \implica Qc.

\subsection{{(I \Y B) \Y V0 = fv}S{fv $<$ V0}}

 La tripla de Hoare {(I \Y B) \Y V0 = fv}S{fv $<$ V0} es válida si y sólo si la expresión $(I \Y B) \Y V0 = fv \implica wp(S,fv $<$ V0)$ es verdadera.

\vspace{0.3cm}

 Donde $wp(S,fv $<$ V0)$ equivale a $wp(if...; i:= i + 1, \longitud{eventos} -i $<$ V0)$, con "if..." representando la elección booleana que ocurre dentro del ciclo.
 Luego, queda: $wp(if..., \longitud{eventos} -i - 1 $<$ V0)$, que equivale a:

\begin{equation}
	def(B_{if}) \yLuego ( B_{if} \Y wp(res:= res*apuesta.c*pago.c , \longitud{eventos} -i - 1 $<$ V0)) \lor
					 (\neg B_{if} \Y wp(res:= res*apuesta.s*pago.s , \longitud{eventos} -i - 1 $<$ V0))
\end{equation}

 Con $B_{if} \equiv eventos[i]$ la guarda del if dentro del ciclo. Entonces la expresión equivale a:

\begin{equation}
\begin{split}
	& 0 $\leq$ i $<$ \longitud{eventos} \yLuego \\
	& (eventos[i] \Y \longitud{eventos} -i - 1 $<$ V0 \lor \neg eventos[i] \Y \longitud{eventos} -i - 1 $<$ V0) \\
	& (\equiv) 0 $\leq$ i $<$ \longitud{eventos} \yLuego \longitud{eventos} -i - 1 $<$ V0
\end{split}
\end{equation}

 Observando la expresión $(I \Y B) \Y V0 = fv$ se nota que esta equivale a:
 
 \begin{equation}
	\begin{split}
		& (0 $\leq$ i $\leq$ \longitud{eventos} \yLuego res = recurso * prodInv)\\
		& \Y V0 = \longitud{eventos} - i\\

		& (\equiv) (0 $\leq$ i $<$ \longitud{eventos} \yLuego\\
		& res = recurso * prodInv)\\
		& \Y V0 = \longitud{eventos} - i
	\end{split}
	\end{equation}
 
 considerando "prodInv" como la productoria del invariante: $\prod\limits_{j=0}^{i-1} \IfThenElse{eventos[j]}{apuesta.c * pago.c}{apuesta.s * pago.s}$.
 Luego la expresión $(I \Y B) \Y V0 = fv \implica wp(S,fv $<$ V0)$ queda como:

\begin{equation}
\begin{split}
	& (0 $\leq$ i $<$ \longitud{eventos} \yLuego res = recurso * prodInv) \Y V0 = \longitud{eventos} - i\\
	& \implica 0 $\leq$ i $<$ \longitud{eventos} \yLuego \longitud{eventos} -i - 1 $<$ V0
\end{split}
\end{equation}

 Como $V0 = \longitud{eventos} - i$ es más fuerte que la expresión $\longitud{eventos} -i - 1 $<$ V0$, la expresión se verifica.

\vspace{0.3cm}

 Luego se concluye que la tripla de Hoare {(I \Y B) \Y V0 = fv}S{fv $<$ V0} es válida.

\subsection{I \Y fv $\leq$ 0 \implica \neg B}

 Observando la primera parte de la expresión a verificar:
 
\begin{equation}
	I \Y fv=0 \equiv I \Y \longitud{eventos} - i = 0 \equiv I \Y i = \longitud{eventos}
\end{equation}

 Luego, como $\neg B \equiv i $\geq$ \longitud{eventos}$, queda:

\begin{equation}
	I \Y i = \longitud{eventos} \implica i $\geq$ \longitud{eventos}
\end{equation}

 Como la espresión $i = \longitud{eventos}$ es más fuerte que $\implica i $\geq$ \longitud{eventos}$,
 la expresión $I \Y fv $\leq$ 0 \implica \neg B$ es verdadera.

 Luego, por lo expuesto en los puntos anteriores, queda demostrado que la implementación dada es correcta respecto de su especificación.

\end{document}
