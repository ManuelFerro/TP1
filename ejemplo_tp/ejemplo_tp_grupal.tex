\documentclass[10pt,a4paper]{article}

\input{AEDmacros}
\usepackage{caratula} % Version modificada para usar las macros de algo1 de ~> https://github.com/bcardiff/dc-tex


\titulo{Trabajo práctico 1: Especificación y WP}
\subtitulo{Fondo monetario común}

\fecha{\today}

\materia{Algoritmos y Estructuras de Datos (ex-algo2)}
\grupo{Grupo --}

\integrante{Chiquinho, Santiago}{490/23}{santichiquinho@gmail.com}
\integrante{Brucellaria, Tomas}{287/23}{tomasbrus54@gmail.com}
\integrante{Caetano, Franco}{351/23}{francocaetano20@gmail.com}
\integrante{Ferro, Manuel}{405/23}{manu.ferro.13@gmail.com}
% Pongan cuantos integrantes quieran

% Declaramos donde van a estar las figuras
% No es obligatorio, pero suele ser comodo
\graphicspath{{../static/}}

\begin{document}

\maketitle

\section{Especificación}
\subsection{redistribucionDeLosFrutos:}

\begin{proc}{nombre}{\In paramIn : \nat, \Inout paramInout : \TLista{\ent}}{tipoRes}
	%    \modifica{parametro1, parametro2,..}
	\requiere{expresionBooleana1}
	\asegura{expresionBooleana2}
	\aux{auxiliar1}{parametros}{tipoRes}{expresion}
	\pred{pred1}{parametros}{expresion} 
\end{proc}

\aux{auxiliarSuelto}{parametros}{tipoRes}{expresion}
% \paraTodo{variable}{tipo}{expresion}
% \existe{variable}{tipo}{expresion}
% Pueden tener [unalinea] para que no se divida en varias lineas
\pred{predSuelto}{parametros}{\paraTodo[unalinea]{variable}{tipo}{algo \implicaLuego expresion}}
\pred{predSuelto}{parametros}{\existe[unalinea]{variable}{tipo}{algo \yLuego expresion}}

\subsection{Problema 2:}

\begin{proc}{trayectoriaDeLosFrutosIndividualesALargoPlazo}
	{\Inout trayectorias: \TLista{\TLista{\float}}, \In cooperan: \TLista{\bool}, \In apuestas: \TLista{\TLista{\float}},
	 \In pagos: \TLista{\TLista{\float}}, \In eventos: \TLista{\TLista{\nat}}}{tipoRes}

	\requiere{
		(\longitud[trayectorias] = \longitud[cooperan] = \longitud[apuestas] = \longitud[pagos] = \longitud[eventos]) \yLuego %chequeo si las listas trabajan el mismo grupo de individuos
		\paraTodo[unalinea]{i,j,k}{\ent}{
			(0 <= i,j < \longitud[pagos] \yLuego 0 <= k < pagos[i]) \implicaLuego %defino bien los enteros para usarlos de índices

			((\longitud[apuestas][i] = \longitud[pagos][i]) \yLuego %chequeo si ambas refieren al mismo grupo de posibles eventos

			(\longitud[eventos][i] = \longitud[eventos][j]) \yLuego %chequeo que cada individuo haya vivido la misma cantidad de eventos

			(pagos[i][k],apuestas[i][k] > 0) \yLuego %chequeo que todos los pagos sean positivos, igual que todas las apuestas

			(\sum\limits_{n=0}^{k} [apuestas][i][n] = 1) \yLuego %chequeo que las apuestas sumadas sean el 100% de los recursos del individuo
			 													 %(esto implica, por el punto anterior, que son todas menores a 1, que representa la totalidad del recurso del individuo)

			(\longitud[trayectorias[i]] = 1) \yLuego (\cab[trayectorias[i]] > 0)) %chequeo que trayectorias esta en "estado incial", y que son todas positivas
			}} %convendría achicar el requiere con auxiliares, uno por cada linea
 
	\asegura{\paraTodo[unalinea]{i}{\ent}{
		(0 <= i < \longitud[eventos]) \implicaLuego %defino el índice
		
		 (\longitud[trayectorias[i]] = \longitud[eventos[i]]) \yLuego %chequeo que al final cada trayectoria mida tanto como tiempo haya transcurrido
	}}%completar el asegura

	\aux{auxiliar1}{parametros}{tipoRes}{expresion}
	\pred{pred1}{parametros}{expresion} 
\end{proc}

\subsection{trayectoriaExtra˜naEscalera:}
% Especificación del programa

\subsection{individuoDecideSiCooperarONo:}
% Especificación del programa

\subsection{individuoActualizaApuesta:}
% Especificación del programa

\section{Demostraciones de correctitud}
%demostrar que el programa frutoDelTrabajoPuramenteIndividual
%es correcto respecto de su espeficifación (en la consigna)
%Se hace con wp
\end{document}
