\documentclass[10pt,a4paper]{article}

\input{AEDmacros}
\usepackage{caratula} % Version modificada para usar las macros de algo1 de ~> https://github.com/bcardiff/dc-tex


\titulo{Trabajo práctico 1: Especificación y WP}
\subtitulo{Fondo monetario común}

\fecha{\today}

\materia{Algoritmos y Estructuras de Datos (ex-algo2)}
\grupo{Grupo BLSSJWSBWWPCUGSLIJKO}

\integrante{Chiquinho, Santiago}{490/23}{santichiquinho@gmail.com}
\integrante{Brucellaria, Tomas}{287/23}{tomasbrus54@gmail.com}
\integrante{Caetano, Franco}{351/23}{francocaetano20@gmail.com}
\integrante{Ferro, Manuel}{405/23}{manu.ferro.13@gmail.com}
% Pongan cuantos integrantes quieran

% Declaramos donde van a estar las figuras
% No es obligatorio, pero suele ser comodo
\graphicspath{{../static/}}

\setcounter{section}{-1}

\begin{document}

\maketitle

\section{Aclaraciones generales}
 La extrañeza del nombre del grupo se debe a que
 este fue generado con la herramienta de generación aleatoria de nombres proveida en el campus
 de la materia.

\begin{itemize}
	\item Los índices de las listas recursos, cooperan, trayectorias, apuestas, pagos, eventos representa el identificador de los
    individuos.
	\item recursos: seq⟨\real⟩ es la lista con el recurso de cada individuo.
	\item cooperan: seq⟨Bool⟩ es la lista que indica T rue si el individuo en dicha posición coopera.
	\item trayectorias: seq⟨seq⟨\real⟩⟩ indica para cada individuo, en cada paso de tiempos, cuántos recursos (\real) cuenta.
	\item eventos: seq⟨seq⟨\nat⟩⟩ indica para cada individuo, en cada paso temporal, qué evento le ha tocado.
	\item apuestas: seq⟨seq⟨\real⟩⟩ indica para cada individuo, para cada evento posible (numerados desde 0), cuánto apostará.
	\item pagos: seq⟨seq⟨\real⟩⟩ indica para cada individuo, para cada evento, cuánto se pagará. Notar que a diferencia del ejemplo, estams resolviendo un caso más general donde el pago de cada evento puede diferir por individuo.
	\item Las personas que no \textit{cooperan} no aportan nada al fondo monetario común.
	\item Los \textit{recursos} iniciales son positivos.
	\item Todos los \textit{pagos} son positivos.
	\item Las \textit{apuestas} de los individuos representan la proporción de los recursos que los individuos invierten a cada una de los eventos posibles. Notar nuevamente que a diferencia del ejemplo, en este caso más general, podríamos tener apuestas distintas para cada evento por cada individuo.
	\item Cada individuo apuesta siempre el mismo porcentaje por cada evento posible (es decir, el mismo número en cada paso temporal). Por ejemplo, si tenemos dos eventos; cara y ceca y apuesta 0, 4 por cara y 0,6 por seca, en cada paso temporal apostará esas proporciones.
    
\end{itemize}

\clearpage 

\section{Especificación}

 Se tienen 5 problemas escritos en lenguaje coloquial, y se pide realizar una especificación correcta ara cada uno de ellos.
 (la correctitud de estas no será verificada, solo se enunciarán las especificiones propuestas).

\subsection{Problema 1:}

\begin{proc}{redistribucionDeLosFrutos:}
	{\In recursos : \TLista{\ent}\nat, \In cooperan : \TLista{\bool}}{\TLista{\ent}}

	\requiere{
		\longitud{recursos} = \longitud{cooperan} \yLuego todoPositivo(recursos)
	}
	\asegura{
		reparticiones(recursos, cooperan) \yLuego todoIgualSiCooperan(recursos, cooperan)
	}

\vspace{0.3cm}

	\pred{todoPositivo}{recursos}
	{\paraTodo[unalinea]{i}{\ent}{
		0 $\leq$ i $<$ \longitud[recursos] \implicaLuego recursos[i] $>$ 0
	}}

	\pred{reparticiones}{recursos, cooperan}
	{\paraTodo[unaLinea]{i}{\ent}{
		(0 $\leq$ i $<$ \longitud[recursos] \implicaLuego ((cooperan[i] = \True \yLuego res[i] = FMC(recursos, cooperan)) \oLuego (cooperan[i] = \False \yLuego res[i] = recursos[i] + FMC(recursos, cooperan))))}
	}

	\pred{todoIgualSiCooperan}{recursos, cooperan}
	{\paraTodo[unaLinea]{i,j}{\ent}{
		0 $\leq$ i,j $<$ \longitud[recursos] \yLuego cooperan[i] = cooperan[j] /implicaLuego recursos[i] = recursos[j]}
	}

	\aux{FMC}{recursos, cooperan}{\nat}
	{\paraTodo[unaLinea]{i}{\ent}{
		0 $\leq$ i $<$ \longitud[recursos] \implicaLuego ( res = (\sum\limits_{n=0}^-{\longitud[recursos] - 1} (if cooperan[i] = \True then recursos[i] else 0 fi) / (\longitud[cooperan] - 1)))}
	}

\end{proc}

\subsection{Problema 2:}

\begin{proc}{trayectoriaDeLosFrutosIndividualesALargoPlazo}
	{\Inout trayectorias: \TLista{\TLista{\float}}, \In cooperan: \TLista{\bool}, \In apuestas: \TLista{\TLista{\float}},
	 \In pagos: \TLista{\TLista{\float}}, \In eventos: \TLista{\TLista{\nat}}}

\vspace{0.3cm}

	\requiere{
		\longitud{trayectorias} = \longitud{cooperan} = \longitud{apuestas} = \longitud{pagos} = \longitud{eventos} \yLuego %chequeo si las listas trabajan el mismo grupo de individuos

		mismoGrupoDeEventos(apuestas, pagos) \yLuego %chequeo si ambas refieren al mismo grupo de posibles eventos

		igualCantidadDeEventosPorIndividuo(eventos) \yLuego %chequeo que cada individuo haya vivido la misma cantidad de eventos

		todoPositivo(apuestas, pagos) \yLuego %chequeo que todos los pagos sean positivos, igual que todas las apuestas

		todoElRecursoFueApostado(apuestas) \yLuego %chequeo que las apuestas sumadas sean el 100% de los recursos del individuo

		trayectoriasInicialesCorrectas(trayectorias)} %chequeo que trayectorias esta en "estado incial", y que son todas positivas

\vspace{0.3cm}

	\asegura{
		PagoApuestaCorrecto(trayectorias, apuestas, pagos, eventos) \yLuego
		PagoPozoCorrecto(trayectorias, cooperan) \yLuego
		\paraTodo[unalinea]{i}{\ent}{
			0 $\leq$ i $<$ \longitud{eventos} \implicaLuego \longitud{trayectorias[i]} = \longitud{eventos[i]} + 1}}
		 	%chequeo que al final cada trayectoria mida 1 más que la lista de eventos

\vspace{0.3cm}

	\pred{todoPositivo}{apuestas, pagos}
	{\paraTodo[unalinea]{i,j}{\ent}{
		(0 $\leq$ i $<$ \longitud{apuestas} \yLuego 0 $\leq$ j $<$ apuestas[i]) \implicaLuego (apuestas[i][j] > 0 \yLuego pagos[i][j] > 0)}}
	
\vspace{0.3cm}
	
		\pred{trayectoriasInicialesCorrectas}{trayectorias}
	{\paraTodo[unalinea]{i}{\ent}{
		(0 $\leq$ i $<$ \longitud{trayectorias}) \implicaLuego (\longitud{trayectorias[i]} = 1) \yLuego (\cab{trayectorias[i]} > 0)}} 
	
\vspace{0.3cm}
	
		\pred{PagoApuestaCorrecto}{trayectorias, apuestas, pagos, eventos}
		{\paraTodo[unalinea]{i,j}{\ent}{(0 $\leq$ i $<$ \longitud{trayectorias}) \yLuego (1 $\leq$ j $<$ \longitud{trayectorias[i]}) \implicaLuego \\ %acomodo los índices
	 		(trayectorias[i][j] = trayectorias[i][j - 1]*apuestas[i][eventos[i][j - 1]]*pagos[i][eventos[i][j - 1]])}}%chequeo que el recurso del i-esimo jugador a tiempo j sea su anterior multiplicado por la apuesta y el pago que correspondan

\vspace{0.3cm}

		\pred{PagoPozoCorrecto}{trayectorias, cooperan}
		{\paraTodo[unalinea]{i,j}{\ent}{(0 $\leq$ i $<$ \longitud{trayectorias}) \yLuego (1 $\leq$ j $<$ \longitud{trayectorias[i]}) \implicaLuego \\
			\IfThenElse{cooperan[i] = True}{trayectorias[i][j] = pagoPozo[i]}{\\trayectorias[i][j] = pagoPozo[i] + trayectorias[i][j-1]}}}%chequeo que el recurso del i-esimo jugador a tiempo j se corresponda con su pago del pozo

\vspace{0.3cm}

	\pred{pagoPozo}{trayectorias, cooperan}
		{\paraTodo[unalinea]{i,j}{\ent}{(0 $\leq$ i $<$ \longitud{trayectorias}) \yLuego (1 $\leq$ j $<$ \longitud{trayectorias[i]}) \implicaLuego \\
		redistribucionDeLosFrutos (recursosATiempoJ(trayectorias, j)[i], cooperan)}}

\vspace{0.3cm}

	\aux{recursosATiempoJ}{trayectorias, j}{\TLista{\float}}
	{\paraTodo[unalinea]{i}{\ent}{
		(0 $\leq$ i $<$ \longitud{trayectorias}) \implicaLuego 
		(res[i] = trayectorias[i][j])}}%arma una lista de los recursos de los jugadores a tiempo j

\end{proc}

\subsection{Problema 3:}

\begin{proc}{trayectoriaExtrañaEscalera:}{\In trayectoria : \TLista{\float}}{\bool}

	\requiere{\longitud{trayectoria} $\geq$ 3} %la función de longitud no printea bien, preguntar
	\asegura{\res = \True \Iff (( \exists ! i \in[1,\longitud{trayectoria}-2]): (trayectoria[i] > trayectoria [i-1] \yLuego trayectoria [i] > trayectoria [i+1] ))}
\end{proc}

\subsection{Problema 4:}

\begin{proc}{individuoDecideSiCooperarONo:}{\In individuo : \nat, \In recursos : \TLista{\ent} ,\Inout cooperan : \TLista{\bool}, \In apuestas : \TLista{\TLista{\ent}}, \In pagos : \TLista{\TLista{\ent}}, \In eventos : \TLista{\TLista{\ent}}}

	\requiere{ (\longitud{recursos}=\longitud{cooperan}=\longitud{apuestas}=\longitud{pagos}=\longitud{eventos})\\}\\
		\requiere{
		(0 $\leq$ individuo $<$ \longitud{cooperan})
        \\}
        \requiere {
		{apuesta}={Apuesta_{0}}\\
		}
        \requiere{
		(\paraTodo[unalinea]{i,j,k,l,m}{\ent}{((0 $\leq$ i $<$ \longitud{recursos} \implicaLuego recursos[i] > 0) \yLuego
		(0 $\leq$ j,m $<$ \longitud{pagos} \implicaLuego pagos[j][m] > 0) \yLuego (0 $\leq$ k,l $<$ \longitud{apuestas} \implicaLuego apuestas[k][l] > 0))})\\}
        \\

\vspace{0.3cm}

	\asegura{
        (\exists ind :seq⟨seq⟨\float⟩⟩)(esApuestaVariante(ind, individuo)) \yLuego\\ (\exists trayectoriaInd:seq⟨seq⟨\real⟩⟩)(esTrayectoria(trayectoriaInd, recursos, eventos, pagos, cooperan, ind)) \yLuego  \\\paraTodo[unalinea]{i}{\ent}{0 $\leq$ i $<$ \longitud{cooperan} \yLuego \paraTodo[unalinea]{A}{\TLista{\TLista{\float}}}{esApuestaVariante(A, i) \yLuego\\ (\forall trayectoriaI : \TLista{\TLista{\float}})(esTrayectoria(trayectoriaI, recursos, eventos, pagos, cooperan,A))}} \implicaLuego\\
		((\IfThenElse{(\exists i : \ent)(0 $\leq$ i $<$ \longitud{cooperan} \wedge cooperan[i] = False)\\}{evaluarDesertores(trayectoriaI, cooperan, apuestas, pagos, eventos)}{0}) \yLuego\\ \IfThenElse{(\exists i : \ent)(0 $\leq$ i $<$ \longitud{cooperan} \wedge cooperan[i] = False)\\}{evaluarCooperadores(trayectoriaI, cooperan, apuestas, pagos, eventos)}{0}) \implicaLuego\\ evaluarCooperacion(individuo, recursos, cooperan, apuestas, pagos, eventos))\\
		}\\

\vspace{0.3cm}

        \aux{evaluarDesertores}{trayectoriaI, cooperan, apuestas, pagos, eventos}{\float}
            {\paraTodo[unalinea]{i}{\ent}{0 $\leq$ i $<$ \longitud{copeeran}  \implicaLuego\\ \frac{(\sum\limits_{i=0}^{\longitud{cooperan} - 1}){(\IfThenElse{cooperan[i] = False}{trayectoriaI[i][\longitud{trayectoriaI} - 1]}{0})}}{(\sum\limits_{i=0}^{\longitud{cooperan} - 1}) (\IfThenElse{cooperan[i]=False}{1}{0})}}}

\vspace{0.3cm}

		\aux{evaluarCooperadores}{trayectoriaI, cooperan, apuestas, pagos, eventos}{\float}
            {\paraTodo[unalinea]{i}{\ent}{0 $\leq$ i $<$ \longitud{copeeran}  \implicaLuego\\ \frac{(\sum\limits_{i=0}^{\longitud{cooperan} - 1}){(\IfThenElse{cooperan[i] = True}{trayectoriaI[i][\longitud{trayectoriaI} - 1]}{0})}}{(\sum\limits_{i=0}^{\longitud{cooperan} - 1}) (\IfThenElse{cooperan[i]=True}{1}{0})}}}
            
\vspace{0.3cm}

        \pred{evaluarCooperacion}{individuo ,recursos ,cooperan ,apuestas   ,pagos ,eventos}{
            (((trayectoriaInd[individuo][\longitud{tryectoria} -1] $\leq$ evaluarDesertores \wedge evaluarDesertores $>$ evaluarCooperadores)\implicaLuego
            (cooperan[individuo]= False)) \oLuego\\ 
            ((trayectoriaInd[individuo][\longitud{tryectoria} -1] $\leq$ evaluarCooperadores \wedge evaluarCooperadores > evaluarDesertores) \implicaLuego
            (cooperan[individuo]=True)) \oLuego ((evaluarDesertores = evaluarRecursosCooperadores) \implicaLuego\\ cooperan[individuo] = cooperan[individuo]))\\
        }
        
\vspace{0.3cm}

        \pred{todoPositivo}{recursos, apuestas, pagos}
	{\paraTodo[unalinea]{i,j}{\ent}{
		(0 $\leq$ i $<$ \longitud{apuestas} \yLuego 0 $\leq$ j $<$ apuestas[i]) \implicaLuego\\
		(apuestas[i][j] > 0 \yLuego pagos[i][j] > 0 \yLuego recursos[i] > 0)}\\
	}
 
\vspace{0.3cm}

	\pred{esTrayectoria}{trayectoria, eventos, individuo, pagos, apuestas,        recursos, cooperan}{\\
		(trayectorias[individuo][0] = recursos[individuo]) \yLuego (\longitud{trayectorias} = \longitud{eventos} + 1) \yLuego\\ PagoPozoCorrecto (trayectorias , cooperan)
	}

\vspace{0.3cm}
        
	\pred{apuestaValida}{A : \TLista{\TLista{\float}}, individuo}{
		(\longitud{A} = \longitud{{Apuesta_{0}}}) \wedge\\ (\paraTodo[unalinea]{i}{\ent}{0 $\leq$ i $<$ \longitud{{Apuesta_{0}}} \wedge i \neq individuo \implicaLuego A[i] = {Apuesta_{0}}[i]}) \wedge 
		\longitud{A[individuo]} = \longitud{{Apuesta_{0}}[0]} \wedge\\ (\paraTodo[unalinea]{j}{\ent}{0 $\leq$ j $<$ \longitud{A[individuo]} \implicaLuego 0 $\leq$ A[individuo[j]] $\leq$ 1})\\
	}

\end{proc}

\subsection{Problema 5:}

\begin{proc}{individuoActualizaApuesta:}
	{\In individuo: \nat, \In recursos: \TLista{\float}, \In cooperan: \TLista{\bool}, \Inout apuestas: \TLista{\TLista{\float}},
	 \In pagos: \TLista{\TLista{\float}}, \In eventos: \TLista{\TLista{\nat}}}

	 \requiere{
		\longitud{recursos} = \longitud{cooperan} = \longitud{apuestas} = \longitud{pagos} = \longitud{eventos}\\ 
	}
	\requiere {
		mismoGrupoDeEventos(apuestas, pagos)\\
	}
	\requiere {
		todoPositivo(recursos, apuestas, pagos)\\
	}
	\requiere {
		todoElRecursoFueApostado(apuestas)\\
	} 	
	\requiere {
		{apuesta}={Apuesta_{0}}\\
	}
	\requiere {
		0 $\leq$ individuo $<$ \longitud{apuestas}\\
	}


	\asegura{\\
		(\exists B :\TLista{\TLista{\float}})(apuestaValida(B,{individuo}) \yLuego\\ (\exists {trayectoriaB}:\TLista{\TLista{\float}})(esTrayectoria(trayectoriaB ,recursos ,eventos ,pagos ,cooperan ,B)) \yLuego  \\
		\paraTodo[unalinea]{A}{\ent}{apuestaValida(A, individuo) \implicaLuego\\ \paraTodo[unalinea]{trayectoria}{\TLista{\TLista{\float}}}{ esTrayectoria(trayectoria ,recursos ,eventos ,pagos ,cooperan, A) \implicaLuego\\
		trayectoriaB[individuo][\longitud{trayectoriaB[individuo]} - 1] $\geq$ trayectoria[individuo][\longitud{trayectoria[individuo]} - 1]}})\\
		}

	\pred{todoPositivo}{recursos, apuestas, pagos}
	{\paraTodo[unalinea]{i,j}{\ent}{
		(0 $\leq$ i $<$ \longitud{apuestas} \yLuego 0 $\leq$ j $<$ apuestas[i]) \implicaLuego\\
		(apuestas[i][j] > 0 \yLuego pagos[i][j] > 0 \yLuego recursos[i] > 0)}\\
	}

	\pred{esTrayectoria}{trayectoria, eventos, individuo, pagos, apuestas, recursos, cooperan}{\\
		(trayectorias[individuo][0] = recursos[individuo]) \yLuego (\longitud{trayectorias} = \longitud{eventos} + 1) \yLuego\\ PagoPozoCorrecto (trayectorias , cooperan)
	}
        
	\pred{apuestaValida}{A : \TLista{\TLista{\float}}, individuo}{\\
		(\longitud{A} = \longitud{{Apuesta_{0}}}) \wedge\\ (\paraTodo[unalinea]{i}{\ent}{0 $\leq$ i $<$ \longitud{{Apuesta_{0}}} \wedge i \neq individuo \implicaLuego A[i] = {Apuesta_{0}}[i]}) \wedge 
		\longitud{A[individuo]} = \longitud{{Apuesta_{0}}[0]} \wedge\\ (\paraTodo[unalinea]{j}{\ent}{0 $\leq$ j $<$ \longitud{A[individuo]} \implicaLuego 0 $\leq$ A[individuo[j]] $\leq$ 1})\\
	}

\end{proc}

\subsection{Auxiliares de Especificación:}

\aux{todoElRecursoFueApostado}{apuestas}{\ent}
	{\paraTodo[unalinea]{i,j}{\ent}{
		(0 $\leq$ i $<$ \longitud{apuestas} \yLuego 0 $\leq$ j $<$ apuestas[i]) \implicaLuego (\sum\limits_{n=0}^{j} [apuestas][i][n])}
	}

\pred{mismoGrupoDeEventos}{apuestas, pagos}
	{\paraTodo[unalinea]{i}{\ent}{
		0 $\leq$ i $<$ \longitud{pagos} \implicaLuego \longitud{apuestas}[i] = \longitud{pagos}[i]}
	}
\pred{igualCantidadDeEventosPorIndividuo}{eventos}
	{\paraTodo[unalinea]{i,j}{\ent}{
		0 $\leq$ i,j $<$ \longitud{eventos} \implicaLuego \longitud{eventos}[i] = \longitud{eventos}[j]}
	}

\section{Demostraciones de correctitud}

 Se tiene la siguiente implementación:
   
\vspace{0.3cm}
   
\begin{lstlisting}
	res := recurso;
	i := 0;
	while (i < eventos.size()) do
		if eventos[i] then
			res := (res * apuesta.c) * pago.c;
		else
			res := (res * apuesta.s) * pago.s;
		endif
		i := i + 1
	endwhile
\end{lstlisting}
   
\vspace{0.3cm}
   
 Y se quiere demostrar que es correcta respecto de la siguiente especificación:
\begin{proc}{frutoDelTrabajoPuramenteIndividual}{\In recurso: \float, \In apuesta: , \In pago: , \In eventos: \TLista{\bool} }{\float}
   
\requiere{apuesta.c + apuesta.s = 1 \land pago.c > 0 \land pago.s > 0 \land apuesta.c > 0 \land apuesta.s > 0 \land recurso > 0}
   
\asegura{$res = recurso * (apuesta.c * pago.c)^{apariciones(eventos,T)} * (apuesta.s * pago.s)^{apariciones(eventos,F)}$}
   
\end{proc}
   
\vspace{0.3cm}
   
 Donde apariciones(eventos,T) es un auxiliar que cuenta cuántas veces aparece el elemento T en la lista eventos.
 Aclarar que se considera que $eventos[i] = True$ equivale a que salió cara.
   
\vspace{0.3cm} 
   
 Para ello, se argumentará con el Teorema del Invariante y el Teorema de Terminación de Ciclos, cuyas hipotesis afirman que
 para ciertos (Pc,Qc) requiere y asegura, y para cierto (while(B)do S) ciclo, I predicado y fv función variante,
 si se llegan a cumplir las siguientes propiedades:
   
\begin{enumerate} \setlength\itemsep{1cm}
	\item Pc \implica I
   
	\item (I \land B)S(I)
   
	\item (I \land \neg B) \implica Qc %chequear si false B printea bien 
   
	\item ((I \land B) \land V0 = fv)S(fv < V0)
   
	\item I \land fv $\leq$ 0 \implica \neg B
\end{enumerate}
   
\vspace{0.3cm}
   
 Entonces I es un invariante del ciclo, se cumple la poscondición a la salida del ciclo,
 la función variante es estrictamente decreciente y si alcanza la cota inferior la guarda ya no se cumple.
 Luego el ciclo es correcto respeto de la especificación y su ejecución siempre termina.
   
\vspace{0.3cm}
	
Se proponen el siguiente Invariante y función variable:
   
\begin{equation}
	I: 0 \leq i \leq \longitud{eventos} \yLuego res = recurso * \prod\limits_{j=0}^{i-1} \IfThenElse{eventos[j]}{apuesta.c * pago.c}{apuesta.s * pago.s}
\end{equation}
   
\begin{equation}
	fv: \longitud{eventos} - i
\end{equation}
   
 Para los datos:
   
\begin{itemize}
	\item Pc : $res = recurso \land i = 0 \land (apuesta.c + apuesta.s = 1 \land pago.c > 0 \land pago.s > 0 \land apuesta.c > 0 \land apuesta.s > 0 \land recurso > 0)$
	\item Qc : $res = recurso * (apuesta.c * pago.c)^{apariciones(eventos, T)} * (apuesta.s * pago.s)^{apariciones(eventos, F)}$
	\item B : i $<$ \longitud{eventos}
	\item S : el ciclo descrito en la implementación.
\end{itemize}
   
 Por motivos de simplificar la notación y garantizar la comprensión del texto, la expresión
 $apuesta.c + apuesta.s = 1 \land pago.c > 0 \land pago.s > 0 \land apuesta.c > 0 \land apuesta.s > 0 \land recurso > 0$
 será referida como Req (abreviasión de requiere) de ahora en adelante.
	
\vspace{0.3cm}
   
 Se comprueba si los dados I y fv cumplen las hipotesis:
   
\subsection{Pc \implica I}
   
\begin{equation}
\begin{split}
	& Pc \implica I \\
	& (\equiv) res = recurso \land i = 0 \land Req  \implica \\
	& 0 \leq i \leq \longitud{eventos} \yLuego res = recurso * \prod\limits_{j=0}^{i-1} \IfThenElse{eventos[j]}{apuesta.c * pago.c}{apuesta.s * pago.s} \\
	& (\equiv) res = recurso \land i = 0 \land Req \implica \\
	& 0 \leq 0 \leq \longitud{eventos} \yLuego res = recurso * \prod\limits_{j=0}^{0-1} \IfThenElse{eventos[j]}{apuesta.c * pago.c}{apuesta.s * pago.s}\\
	& (\equiv) res = recurso \land i = 0 \land Req \implica true \yLuego res = recurso * 1
\end{split}
\end{equation}
   
 Luego, se comprueba que la precondición es más fuerte que el invariante.
   
 \subsection{(I \land B)S(I)}
   
 La tripla de Hoare $(I \land B)S(I)$ es válida si y sólo si la expresión
 $I \land B \implica wp(S,I)$ es verdadera.
   
\vspace{0.3cm}
   
 Donde $wp(S,I)$ equivale a $wp(if...; i:= i + 1, I)$, con if... representando la elección booleana que ocurre dentro del ciclo.
 Luego, queda: $wp(if..., I^{i}_{i+1})$, que equivale a:
   
\begin{equation}
\begin{split}
	& def(B_{if}) \yLuego ( B_{if} \land wp(res:= res*apuesta.c*pago.c , I^{i}_{i+1})) \lor 
	(\neg B_{if} \land wp(res:= res*apuesta.s*pago.s , I^{i}_{i+1}))\\
	& (\equiv) def(B_{if}) \yLuego (wp1 \lor wp2)
\end{split}
\end{equation}
   
 Con $B_{if} \equiv eventos[i]$ la guarda del if dentro del ciclo y 
 wp1; wp2 abreviaciones de las 2 expresiones de wp. 
 Observar que $B_{if}$ es la misma guarda dentro de la productoria,
 por lo tanto si separamos en casos cuando $B_{if} \lor \neg B_{if}$,
 se puede sacar de la productoria el caso i-ésimo dependiendo de en que caso estemos.
   
\vspace{0.3cm}
	
 (Analizando ambos casos, omitiremos la indexación de i del invariante, que al reemplazar por i+1 es igual a\\
 $-1 \leq i \leq \longitud{eventos} - 1$).
 Observando wp1:
   
\begin{equation}
\begin{split}
	& res * (apuesta.c * pago.c) =\\
	& recurso * \prod\limits_{j=0}^{i} \IfThenElse{eventos[j]}{apuesta.c * pago.c}{apuesta.s * pago.s} =\\
	& recurso * \prod\limits_{j=0}^{i-1} \IfThenElse{eventos[j]}{apuesta.c * pago.c}{apuesta.s * pago.s} * (apuesta.c * pago.c) 
\end{split}
\end{equation}
   
 Como apuesta.c y pago.c no pueden ser nulos (el requiere de la especificación admite apuestas y pagos mayores a cero), podemos pasar dividiendo y queda:
   
\begin{equation}
	res = recurso * \prod\limits_{j=0}^{i-1} \IfThenElse{eventos[j]}{apuesta.c * pago.c}{apuesta.s * pago.s}
\end{equation}
   
 Análogamente, como apuesta.s y pago.s son no nulos, wp2 equivale a:
   
\begin{equation}
\begin{split}
	& res * (apuesta.s * pago.s)\\
	& = recurso * \prod\limits_{j=0}^{i} \IfThenElse{eventos[j]}{apuesta.c * pago.c}{apuesta.s * pago.s}\\
	& = recurso * \prod\limits_{j=0}^{i-1} \IfThenElse{eventos[j]}{apuesta.c * pago.c}{apuesta.s * pago.s} * (apuesta.s * pago.s)\\
\end{split}
\end{equation}
   
 Luego:
   
\begin{equation}
	res = recurso * \prod\limits_{j=0}^{i-1} \IfThenElse{eventos[j]}{apuesta.c * pago.c}{apuesta.s * pago.s}
\end{equation}
   
 Entonces $wp(S,I)$ equivale a:
   
\begin{equation}
\begin{split}
	& 0 \leq i < \longitud{eventos} \yLuego\\
	& (( B_{if} \land -1 \leq i \leq \longitud{eventos} - 1 \land res = recurso * \prod\limits_{j=0}^{i-1} \IfThenElse{eventos[j]}{apuesta.c * pago.c}{apuesta.s * pago.s} ) \lor\\
   
	& ( \neg B_{if} \land -1 \leq i \leq \longitud{eventos} - 1 \land res = recurso * \prod\limits_{j=0}^{i-1} \IfThenElse{eventos[j]}{apuesta.c * pago.c}{apuesta.s * pago.s} ))
\end{split}
\end{equation}
	
 Notar que dejamos de omitir la indexación de i del invariante, y $def(B_{if}) = 0 \leq i < \longitud{eventos}$
 Simplificando la ecuación, nos queda:
   
\begin{equation}
\begin{split}
	& 0 \leq i < \longitud{eventos} \land res = recurso * \prod\limits_{j=0}^{i-1} \IfThenElse{eventos[j]}{apuesta.c * pago.c}{apuesta.s * pago.s} \yLuego (eventos[i] = True \lor eventos[i] = False)\\
   
	& 0 \leq i < \longitud{eventos} \land res = recurso * \prod\limits_{j=0}^{i-1} \IfThenElse{eventos[j]}{apuesta.c * pago.c}{apuesta.s * pago.s}
\end{split}
\end{equation}
   
 Observando la expresión $I \land B$, recordando que $B \equiv i < \longitud{eventos}$ se obtiene que equivale a:
   
\begin{equation}
	0 \leq i < \longitud{eventos} \yLuego res = recurso * \prod\limits_{j=0}^{i-1} \IfThenElse{eventos[j]}{apuesta.c * pago.c}{apuesta.s * pago.s}
\end{equation}
   
 Por lo tanto nos queda la misma ecuación en ambos lados de la implicación:
   
\begin{equation}
\begin{split}
	& I \land B \implica wp(S,I)\\
	& (\equiv) 0 \leq i < \longitud{eventos} \yLuego res = recurso * \prod\limits_{j=0}^{i-1} \IfThenElse{eventos[j]}{apuesta.c * pago.c}{apuesta.s * pago.s}\\
	&\implica 0 \leq i < \longitud{eventos} \yLuego res = recurso * \prod\limits_{j=0}^{i-1} \IfThenElse{eventos[j]}{apuesta.c * pago.c}{apuesta.s * pago.s}
\end{split}
\end{equation}
   
 Y terminamos de confirmar que $I$ es un invariante durante del ciclo.
   
\subsection{(I \land \neg B) \implica Qc}
   
 Como la expresión $\neg B$ equivale a $i \geq \longitud{eventos}$, la expresión $I \land \neg B$ equivale a:
	
\begin{equation}
\begin{split}
	& \leq i = \longitud{eventos} \yLuego
 res = recurso * \prod\limits_{j=0}^{i-1} \IfThenElse{eventos[j]}{apuesta.c * pago.c}{apuesta.s * pago.s}\\
   
	& (\equiv) res = recurso * \prod\limits_{j=0}^{\longitud{eventos}-1} \IfThenElse{eventos[j]}{apuesta.c * pago.c}{apuesta.s * pago.s}\\
\end{split}
\end{equation}
   
 Observar que j indexa perfectamente a $eventos$. La productoria multiplica el término (apuesta.c * pago.c) las veces que se cumpla $ eventos[j] = True$,
 es decir, el término (apuesta.c * pago.c) aparece multiplicando exactamente (eventos, T). Luego,
 (apuesta.c * pago.c)$^{apariciones(eventos, T)}$ es una parte de la productoria.
   
\vspace{0.3cm}
	 
 Analogamente, (apuesta.s * pago.s)$^{apariciones(eventos, F)}$ es una parte de la productoria. como la productoria
 opera con un if cuya guarda es booleana, los términos de la misma solo se comportan de 2 maneras, las descritas anteriormente.
 Luego:
   
\begin{equation}
\begin{split}
	& res = recurso * \prod\limits_{j=0}^{\longitud{eventos}-1} \IfThenElse{eventos[j]}{apuesta.c * pago.c}{apuesta.s * pago.s}\\
	& (\equiv) res = recurso * (apuesta.c * pago.c)^{apariciones(eventos, T)} * (apuesta.s * pago.s)^{apariciones(eventos, F)}
\end{split}
\end{equation}
   
 Al ser la segunda expresión exactamente Qc, se concluye que (I \land \neg B) \implica Qc.
   
\subsection{((I \land B) \land V0 = fv)S(fv < V0)}
   
 La tripla de Hoare $((I \land B) \land V0 = fv)S(fv < V0)$ es válida si y sólo si la expresión
 $(I \land B) \land V0 = fv \implica wp(S,fv < V0)$ es verdadera.
   
\vspace{0.3cm}
   
 Donde $wp(S,fv < V0)$ equivale a $wp(if...; i:= i + 1, \longitud{eventos} -i < V0)$,
 con if... representando la elección booleana que ocurre dentro del ciclo.
 Luego, queda: $wp(if..., \longitud{eventos} -i - 1 < V0)$, que equivale a:
   
\begin{equation}
\begin{split}
	& def(B_{if}) \yLuego\\
	& ( B_{if} \land wp(res:= res*apuesta.c*pago.c , \longitud{eventos} -i - 1 < V0)) \lor\\
	& (\neg B_{if} \land wp(res:= res*apuesta.s*pago.s , \longitud{eventos} -i - 1 < V0))
\end{split}
\end{equation}
   
 Con $B_{if} \equiv eventos[i]$ la guarda del if dentro del ciclo. Entonces la expresión equivale a:
   
\begin{equation}
\begin{split}
	& 0 \leq i < \longitud{eventos} \yLuego \\
	& (eventos[i] \land \longitud{eventos} -i - 1 < V0 \lor \neg eventos[i] \land \longitud{eventos} -i - 1 < V0) \\
	& (\equiv) 0 \leq i < \longitud{eventos} \yLuego \longitud{eventos} -i - 1 < V0
\end{split}
\end{equation}
   
 Observando la expresión $(I \land B) \land V0 = fv$ se nota que esta equivale a:
	
\begin{equation}
\begin{split}
	& (0 \leq i \leq \longitud{eventos} \yLuego res = recurso * prodInv) \land V0 = \longitud{eventos} - i \\
	& (\equiv) (0 \leq i < \longitud{eventos} \yLuego res = recurso * prodInv) \land V0 = \longitud{eventos} - i \\
\end{split}
\end{equation}
	
 considerando prodInv como la productoria del invariante: $\prod\limits_{j=0}^{i-1} \IfThenElse{eventos[j]}{apuesta.c * pago.c}{apuesta.s * pago.s}$.
 Luego la expresión $(I \land B) \land V0 = fv \implica wp(S,fv < V0)$ queda como:
   
\begin{equation}
\begin{split}
	& (0 \leq i < \longitud{eventos} \yLuego res = recurso * prodInv) \land V0 = \longitud{eventos} - i\\
	& \implica 0 \leq i < \longitud{eventos} \yLuego \longitud{eventos} -i - 1 < V0
\end{split}
\end{equation}
   
 Como $V0 = \longitud{eventos} - i$ es más fuerte que la expresión $\longitud{eventos} -i - 1 < V0$, la expresión se verifica.
   
\vspace{0.3cm}
   
 Luego, se concluye que la tripla de Hoare $((I \land B) \land V0 = fv)S(fv < V0)$ es válida, lo que implica
 que la función variante fv es estrictamente decreciente.
   
\subsection{I \land fv $\leq$ 0 \implica \neg B}
   
 Observando la primera parte de la expresión a verificar:
	
\begin{equation}
	I \land fv=0 \equiv I \land \longitud{eventos} - i = 0 \equiv I \land i = \longitud{eventos}
\end{equation}
   
 Luego, como $\neg B \equiv i \geq \longitud{eventos}$, queda:
   
\begin{equation}
	I \land i = \longitud{eventos} \implica i \geq \longitud{eventos}
\end{equation}
   
 Como la espresión $i = \longitud{eventos}$ es más fuerte que $\implica i \geq \longitud{eventos}$,
 la expresión $I \land fv \leq 0 \implica \neg B$ es verdadera.
   
\vspace{0.3cm}
  
 Luego, por lo expuesto en los puntos anteriores, queda demostrado que la implementación dada es correcta respecto de su especificación.
  
\end{document}
   