\documentclass[10pt,a4paper]{article}

\usepackage[spanish,activeacute,es-tabla]{babel}
\usepackage[utf8]{inputenc}
\usepackage{ifthen}
\usepackage{listings}
\usepackage{dsfont}
\usepackage{subcaption}
\usepackage{amsmath}
\usepackage[strict]{changepage}
\usepackage[top=1cm,bottom=2cm,left=1cm,right=1cm]{geometry}%
\usepackage{color}%
\newcommand{\tocarEspacios}{%
	\addtolength{\leftskip}{3em}%
	\setlength{\parindent}{0em}%
}

% Especificacion de procs

\newcommand{\In}{\textsf{in }}
\newcommand{\Out}{\textsf{out }}
\newcommand{\Inout}{\textsf{inout }}

\newcommand{\encabezadoDeProc}[4]{%
	% Ponemos la palabrita problema en tt
	%  \noindent%
	{\normalfont\bfseries\ttfamily proc}%
	% Ponemos el nombre del problema
	\ %
	{\normalfont\ttfamily #2}%
	\
	% Ponemos los parametros
	(#3)%
	\ifthenelse{\equal{#4}{}}{}{%
		% Por ultimo, va el tipo del resultado
		\ : #4}
}

\newenvironment{proc}[4][res]{%
	
	% El parametro 1 (opcional) es el nombre del resultado
	% El parametro 2 es el nombre del problema
	% El parametro 3 son los parametros
	% El parametro 4 es el tipo del resultado
	% Preambulo del ambiente problema
	% Tenemos que definir los comandos requiere, asegura, modifica y aux
	\newcommand{\requiere}[2][]{%
		{\normalfont\bfseries\ttfamily requiere}%
		\ifthenelse{\equal{##1}{}}{}{\ {\normalfont\ttfamily ##1} :}\ %
		\{\ensuremath{##2}\}%
		{\normalfont\bfseries\,\par}%
	}
	\newcommand{\asegura}[2][]{%
		{\normalfont\bfseries\ttfamily asegura}%
		\ifthenelse{\equal{##1}{}}{}{\ {\normalfont\ttfamily ##1} :}\
		\{\ensuremath{##2}\}%
		{\normalfont\bfseries\,\par}%
	}
	\renewcommand{\aux}[4]{%
		{\normalfont\bfseries\ttfamily aux\ }%
		{\normalfont\ttfamily ##1}%
		\ifthenelse{\equal{##2}{}}{}{\ (##2)}\ : ##3\, = \ensuremath{##4}%
		{\normalfont\bfseries\,;\par}%
	}
	\renewcommand{\pred}[3]{%
		{\normalfont\bfseries\ttfamily pred }%
		{\normalfont\ttfamily ##1}%
		\ifthenelse{\equal{##2}{}}{}{\ (##2) }%
		\{%
		\begin{adjustwidth}{+5em}{}
			\ensuremath{##3}
		\end{adjustwidth}
		\}%
		{\normalfont\bfseries\,\par}%
	}
	
	\newcommand{\res}{#1}
	\vspace{1ex}
	\noindent
	\encabezadoDeProc{#1}{#2}{#3}{#4}
	% Abrimos la llave
	\par%
	\tocarEspacios
}
{
	% Cerramos la llave
	\vspace{1ex}
}

\newcommand{\aux}[4]{%
	{\normalfont\bfseries\ttfamily\noindent aux\ }%
	{\normalfont\ttfamily #1}%
	\ifthenelse{\equal{#2}{}}{}{\ (#2)}\ : #3\, = \ensuremath{#4}%
	{\normalfont\bfseries\,;\par}%
}

\newcommand{\pred}[3]{%
	{\normalfont\bfseries\ttfamily\noindent pred }%
	{\normalfont\ttfamily #1}%
	\ifthenelse{\equal{#2}{}}{}{\ (#2) }%
	\{%
	\begin{adjustwidth}{+2em}{}
		\ensuremath{#3}
	\end{adjustwidth}
	\}%
	{\normalfont\bfseries\,\par}%
}

% Tipos

\newcommand{\nat}{\ensuremath{\mathds{N}}}
\newcommand{\ent}{\ensuremath{\mathds{Z}}}
\newcommand{\float}{\ensuremath{\mathds{R}}}
\newcommand{\bool}{\ensuremath{\mathsf{Bool}}}
\newcommand{\cha}{\ensuremath{\mathsf{Char}}}
\newcommand{\str}{\ensuremath{\mathsf{String}}}

% Logica

\newcommand{\True}{\ensuremath{\mathrm{true}}}
\newcommand{\False}{\ensuremath{\mathrm{false}}}
\newcommand{\Then}{\ensuremath{\rightarrow}}
\newcommand{\Iff}{\ensuremath{\leftrightarrow}}
\newcommand{\implica}{\ensuremath{\longrightarrow}}
\newcommand{\IfThenElse}[3]{\ensuremath{\mathsf{if}\ #1\ \mathsf{then}\ #2\ \mathsf{else}\ #3\ \mathsf{fi}}}
\newcommand{\yLuego}{\land _L}
\newcommand{\oLuego}{\lor _L}
\newcommand{\implicaLuego}{\implica _L}

\newcommand{\cuantificador}[5]{%
	\ensuremath{(#2 #3: #4)\ (%
		\ifthenelse{\equal{#1}{unalinea}}{
			#5
		}{
			$ % exiting math mode
			\begin{adjustwidth}{+2em}{}
				$#5$%
			\end{adjustwidth}%
			$ % entering math mode
		}
		)}
}

\newcommand{\existe}[4][]{%
	\cuantificador{#1}{\exists}{#2}{#3}{#4}
}
\newcommand{\paraTodo}[4][]{%
	\cuantificador{#1}{\forall}{#2}{#3}{#4}
}

%listas

\newcommand{\TLista}[1]{\ensuremath{seq \langle #1\rangle}}
\newcommand{\lvacia}{\ensuremath{[\ ]}}
\newcommand{\lv}{\ensuremath{[\ ]}}
\newcommand{\longitud}[1]{\ensuremath{|#1|}}
\newcommand{\cons}[1]{\ensuremath{\mathsf{addFirst}}(#1)}
\newcommand{\indice}[1]{\ensuremath{\mathsf{indice}}(#1)}
\newcommand{\conc}[1]{\ensuremath{\mathsf{concat}}(#1)}
\newcommand{\cab}[1]{\ensuremath{\mathsf{head}}(#1)}
\newcommand{\cola}[1]{\ensuremath{\mathsf{tail}}(#1)}
\newcommand{\sub}[1]{\ensuremath{\mathsf{subseq}}(#1)}
\newcommand{\en}[1]{\ensuremath{\mathsf{en}}(#1)}
\newcommand{\cuenta}[2]{\mathsf{cuenta}\ensuremath{(#1, #2)}}
\newcommand{\suma}[1]{\mathsf{suma}(#1)}
\newcommand{\twodots}{\ensuremath{\mathrm{..}}}
\newcommand{\masmas}{\ensuremath{++}}
\newcommand{\matriz}[1]{\TLista{\TLista{#1}}}
\newcommand{\seqchar}{\TLista{\cha}}

\renewcommand{\lstlistingname}{Código}
\lstset{% general command to set parameter(s)
	language=Java,
	morekeywords={endif, endwhile, skip},
	basewidth={0.47em,0.40em},
	columns=fixed, fontadjust, resetmargins, xrightmargin=5pt, xleftmargin=15pt,
	flexiblecolumns=false, tabsize=4, breaklines, breakatwhitespace=false, extendedchars=true,
	numbers=left, numberstyle=\tiny, stepnumber=1, numbersep=9pt,
	frame=l, framesep=3pt,
	captionpos=b,
}

\usepackage{caratula} % Version modificada para usar las macros de algo1 de ~> https://github.com/bcardiff/dc-tex


\titulo{Trabajo práctico 1: Especificación y WP}
\subtitulo{Fondo monetario común}

\fecha{\today}

\materia{Algoritmos y Estructuras de Datos (ex-algo2)}
\grupo{Grupo --}

\integrante{Chiquinho, Santiago}{490/23}{santichiquinho@gmail.com}
\integrante{Brucellaria, Tomas}{287/23}{tomasbrus54@gmail.com}
\integrante{Caetano, Franco}{351/23}{francocaetano20@gmail.com}
\integrante{Ferro, Manuel}{405/23}{manu.ferro.13@gmail.com}
% Pongan cuantos integrantes quieran

% Declaramos donde van a estar las figuras
% No es obligatorio, pero suele ser comodo
\graphicspath{{../static/}}

\begin{document}

\maketitle

\section{Especificación}

\subsection{Problema 1:}

\begin{proc}{redistribucionDeLosFrutos:}{\In paramIn : \nat, \Inout paramInout : \TLista{\ent}}{tipoRes}
	%    \modifica{parametro1, parametro2,..}
	\requiere{expresionBooleana1}
	\asegura{expresionBooleana2}
	\aux{auxiliar1}{parametros}{tipoRes}{expresion}
	\pred{pred1}{parametros}{expresion} 
\end{proc}

\subsection{Problema 2:}

\begin{proc}{trayectoriaDeLosFrutosIndividualesALargoPlazo}
	{\Inout trayectorias: \TLista{\TLista{\float}}, \In cooperan: \TLista{\bool}, \In apuestas: \TLista{\TLista{\float}},
	 \In pagos: \TLista{\TLista{\float}}, \In eventos: \TLista{\TLista{\nat}}}

	\requiere{
		\longitud[trayectorias] = \longitud[cooperan] = \longitud[apuestas] = \longitud[pagos] = \longitud[eventos] \yLuego %chequeo si las listas trabajan el mismo grupo de individuos

		mismoGrupoDeEventos(apuestas, pagos) \yLuego %chequeo si ambas refieren al mismo grupo de posibles eventos

		igualCantidadDeEventosPorIndividuo(eventos) \yLuego %chequeo que cada individuo haya vivido la misma cantidad de eventos

		todoPositivo(apuestas, pagos) \yLuego %chequeo que todos los pagos sean positivos, igual que todas las apuestas

		todoElRecursoFueApostado(apuestas) \yLuego %chequeo que las apuestas sumadas sean el 100% de los recursos del individuo

		trayectoriasInicialesCorrectas(trayectorias) %chequeo que trayectorias esta en "estado incial", y que son todas positivas
		}
 
	\asegura{
		PagoApuestaCorrecto(trayectorias, apuestas, pagos, eventos) \yLuego
		PagoPozoCorrecto(trayectorias, cooperan) \yLuego
		\paraTodo[unalinea]{i}{\ent}{
			0 $\leq$ i $<$ \longitud[eventos] \implicaLuego \longitud[trayectorias[i]] = \longitud[eventos[i]] + 1
		 	%chequeo que al final cada trayectoria mida 1 más que la lista de eventos
		}}

	\pred{todoPositivo}{apuestas, pagos}
	{\paraTodo[unalinea]{i,j}{\ent}{
		(0 $\leq$ i $<$ \longitud[apuestas] \yLuego 0 $\leq$ j $<$ apuestas[i]) \implicaLuego (apuestas[i][j] $>$ 0 \yLuego pagos[i][j] $>$ 0)}
	}
	
	\pred{trayectoriasInicialesCorrectas}{trayectorias}
	{\paraTodo[unalinea]{i}{\ent}{
		(0 $\leq$ i $<$ \longitud[trayectorias]) \implicaLuego ((\longitud[trayectorias[i]] = 1) \yLuego ((\cab)[trayectorias[i]] $>$ 0))}
	} 
	
	\pred{PagoApuestaCorrecto}{trayectorias, apuestas, pagos, eventos}
		{\paraTodo[unalinea]{i,j}{\ent}{
			((0 $\leq$ i $<$ \longitud[trayectorias]) \yLuego (1 $\leq$ j $<$ \longitud[trayectorias[i]])) \implicaLuego %acomodo los índices
	 		(trayectorias[i][j] = trayectorias[i][j - 1]*apuestas[i][eventos[i][j - 1]]*pagos[i][eventos[i][j - 1]])
	 	}
		}%chequeo que el recurso del i-esimo jugador a tiempo j sea su anterior multiplicado por la apuesta y el pago que correspondan

	\pred{PagoPozoCorrecto}{trayectorias, cooperan}
		{\paraTodo[unalinea]{i,j}{\ent}{ ((0 $\leq$ i $<$ \longitud[trayectorias]) \yLuego (1 $\leq$ j $<$ \longitud[trayectorias[i]])) \implicaLuego
			\IfThenElse{cooperan[i] = True}
				{trayectorias[i][j] = redistribucionDeLosFrutos (recursosATiempoJ(trayectorias, j)[i], cooperan)[i]}
				{trayectorias[i][j] = redistribucionDeLosFrutos (recursosATiempoJ(trayectorias, j)[i], cooperan)[i] + trayectorias[i][j - 1]}
		}
		}%chequeo que el recurso del i-esimo jugador a tiempo j se corresponda con su pago del pozo

	\aux{recursosATiempoJ}{trayectorias, j}{\TLista{\float}}
	{\paraTodo[unalinea]{i}{\ent}{
		(0 $\leq$ i $<$ \longitud[trayectorias]) \implicaLuego 
		(res[i] = trayectorias[i][j]) 
	}}%arma una lista de los recursos de los jugadores a tiempo j

\end{proc}

\subsection{Problema 3:}

\begin{proc}{trayectoriaExtrañaEscalera:}{\In trayectoria : \TLista{\float}}{\bool}

	\requiere{\longitud[trayectoria] $\geq$ 3} %la función de longitud no printea bien, preguntar
	\asegura{\res = \True \Iff (( \exists ! i \in[1,\longitud[trayectoria]-2]): (trayectoria[i] > trayectoria [i-1] \yLuego trayectoria [i] > trayectoria [i+1] ))}
\end{proc}

\subsection{Problema 4:}

\begin{proc}{individuoDecideSiCooperarONo:}{\In paramIn : \nat, \Inout paramInout : \TLista{\ent}}{tipoRes}

	\requiere{expresionBooleana1}
	\asegura{expresionBooleana2}
	\aux{auxiliar1}{parametros}{tipoRes}{expresion}
	\pred{pred1}{parametros}{expresion} 
\end{proc}

\subsection{Problema 5:}

\begin{proc}{individuoActualizaApuesta:}
	{\In individuo: \nat, \In recursos: \TLista{\float}, \In cooperan: \TLista{\bool}, \Inout apuestas: \TLista{\TLista{\float}},
	 \In pagos: \TLista{\TLista{\float}}, \In eventos: \TLista{\TLista{\nat}}}

	\requiere{
		\longitud[recursos] = \longitud[cooperan] = \longitud[apuestas] = \longitud[pagos] = \longitud[eventos] \yLuego

		mismoGrupoDeEventos(apuestas, pagos) \yLuego

		todoPositivo(recursos, apuestas, pagos) \yLuego

		todoElRecursoFueApostado(apuestas) \yLuego	

		0 $\leq$ individuo $<$ \longitud[apuestas]}
		
	\asegura{%¿habrá que chequear que solo cambió apuestas[individuo]? ¿cómo se haría eso?

		\longitud[apuestas] = \longitud[pagos] \yLuego %chequeo que la lista de apuestas conserve la longitud

		%la longitud podría chequearse con pagos, eventos, cooperan o recursos,
		%¿es correcto exigir que se chequee con pagos? sino,
		%¿habría que hacer un auxiliar para que se permita chequear la longitud con cualquiera? preguntar
		
		mismoGrupoDeEventos(apuestas, pagos) \yLuego %chequeo que todas las apuestas (en particular apuestas[individuo]) conserven la longitud

		todoPositivo(recursos, apuestas, pagos) \yLuego %reciclo el pred ya definido, pero este chequea la condición también en recursos y pagos,
														%¿es correcto? ¿o deberìa chequearse sólo apuestas?
		
		todoElRecursoFueApostado(apuestas) \yLuego %chequeo que todas las apuestas (en particular apuestas[individuo]) sigan sumando 1
		 
		RecursosFinales(individuo, recursos, apuestas, pagos, eventos) $\geq$ RecursosFinales(individuo, recursos, apuestas, pagos, eventos)
		%la idea es chequear que los recursos finales individuales que las nuevas apuestas generan sean "los máximos"
		%¿Cómo hacerlo?
		
		}%se supone que apuestas ya está modificado, sino es así preguntar cómo se indica eso

	\pred{todoPositivo}{recursos, apuestas, pagos}
	{\paraTodo[unalinea]{i,j}{\ent}{
		(0 $\leq$ i $<$ \longitud[apuestas] \yLuego 0 $\leq$ j $<$ apuestas[i]) \implicaLuego
		(apuestas[i][j] $>$ 0 \yLuego pagos[i][j] $>$ 0 \yLuego recursos[i] $>$ 0)}
	}

	\aux{RecursosFinales}{individuo, recursos, apuestas, pagos, eventos}{\float}
	{recursos[individuo]*(\prod_{n=0}^{\longitud[eventos[individuo]] - 1} [pagos][individuo][eventos[individuo][n]]*[apuestas][individuo][eventos[individuo][n]])}
	%una productoria que calcula todo el proceso multiplicativo

\end{proc}

\subsection{Auxiliares de Especificación:}

\aux{todoElRecursoFueApostado}{apuestas}{\ent}
	{\paraTodo[unalinea]{i,j}{\ent}{
		(0 $\leq$ i $<$ \longitud[apuestas] \yLuego 0 $\leq$ j $<$ apuestas[i]) \implicaLuego (\sum\limits_{n=0}^{j} [apuestas][i][n])}
	}

\pred{mismoGrupoDeEventos}{apuestas, pagos}
	{\paraTodo[unalinea]{i}{\ent}{
		0 $\leq$ i $<$ \longitud[pagos] \implicaLuego \longitud[apuestas][i] = \longitud[pagos][i]}
	}
\pred{igualCantidadDeEventosPorIndividuo}{eventos}
	{\paraTodo[unalinea]{i,j}{\ent}{
		0 $\leq$ i,j $<$ \longitud[eventos] \implicaLuego \longitud[eventos][i] = \longitud[eventos][j]}
	}

\section{Demostraciones de correctitud}
%demostrar que el programa frutoDelTrabajoPuramenteIndividual
%es correcto respecto de su espeficifación (en la consigna)
%Se hace con wp
\end{document}
